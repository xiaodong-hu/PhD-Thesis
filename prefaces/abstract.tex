\chapter*{Abstract}

Twisted two-dimensional (2D) van der Waals (vdW) materials have become a fertile ground for discovering novel quantum phenomena. The stacking of layers and twisting angles serve as independent tuning parameters, expanding the search space for quantum phases of matter for both experimentalist and theorists. Additionally, the formation of moiré superlattices, as a unique property of twisted vdW materials, is able to cover a wide range of energy and length scales controlled by the twisting angles. This makes such platform an efficient playground for exploring exotic quantum phases. This thesis is divided into two parts based on whether moiré physics is involved.

In the first part, we focus on the fractional Chern insulators (FCI) realized on the moir\'e miniband of twisted transition metal dichalchogenides. FCI were proposed theoretically about a decade ago. These exotic states of matter are fractional quantum Hall states realized when a nearly flat Chern band is partially filled, even in the absence of an external magnetic field. Recently, exciting experimental signatures of such states have been reported in twisted MoTe$_2$ bilayer systems. Motivated by these experimental and theoretical progresses, in this paper, we develop a projective construction for the composite fermion states (either the  Jain's sequence or the composite Fermi liquid) in a partially filled Chern band with Chern number $C=\pm1$, which is capable of capturing the microscopics, e.g., symmetry fractionalization patterns and magnetoroton excitations. On the mean-field level, the ground states' and excitated states' composite fermion wavefunctions are found self-consistently in an enlarged Hilbert space. Beyond the mean-field, these wavefunctions can be projected back to the physical Hilbert space to construct the electronic wavefunctions, allowing direct comparison with FCI states from exact diagonalization on finite lattices. We find that the projected electronic wavefunction corresponds to the \emph{combinatorial hyperdeterminant} of a tensor. When applied to the traditional Galilean invariant Landau level context, the present construction exactly reproduces Jain's composite fermion wavefunctions. We apply this projective construction to the twisted bilayer MoTe$_2$ system. Experimentally relevant properties are computed, such as the magnetoroton band structures and quantum numbers.

In the second part, we focus on the platform of twisted 2D superconductors where moir\'e physics is absent. We first consider the twisted Ising superconductors like NbSe$_2$ and TaS$_2$. These materials have demonstrated Ising superconductivity down to atomically thin layers. Due to the spin-orbit coupling, these superconductors have the in-plane upper critical magnetic field far beyond the Pauli limit. We theoretically demonstrate that, twisted bilayer Ising superconductors separated by a ferromagnetic buffer layer can naturally host chiral topological superconductivity with Chern numbers, which can be realized in heterostructures like $\mathrm{NbSe_2/CrCl_3/NbSe_2}$. Under appropriate experimental conditions the topological superconducting gap can reach $>0.1$ meV, leading to readily observable signatures such as the quantized thermal Hall transport at low temperatures. Then we take twisted 2D superconductors as a general platform for transport measurement study. 2D superconductors have been realized in various atomically thin films such as the twisted bilayer graphene, some of which are anticipated to involve unconventional pairing mechanism. Due to their low dimensionality, experimental probes of the exact nature of superconductivity in these systems have been limited. We propose, by applying a \emph{vertical} supercurrent to a bilayer superconductor where the mirror symmetry is naturally broken by the twisting, there will be anomalous thermal Hall effect induced by the supercurrent that can serve as a sharp probe for the \emph{in-plane} anisotropy of the superconducting gap function. This effect occurs in the \emph{absence} of an external magnetic field and spontaneous breaking of the time-reversal symmetry in the ground state. We derive explicit formulas for the induced thermal Hall conductivity and show them to be significant in the examples of twisted cuprates and twisted FeSe where monolayer superconductivity have already been observed. Though technical challenges still exist, we propose this to be a generic probe of the gap anisotropy in a twisted bilayer superconductor.